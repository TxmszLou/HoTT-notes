\documentclass{article}

\usepackage{amsmath,amssymb,amsthm,mathtools,paralist,bm,mathrsfs}
\usepackage{
  nameref, % \nameref
  hyperref, % \autoref
  cleveref, % \cref
}

\usepackage{notation/modes}

\usepackage[french,english]{babel}
\usepackage[autostyle]{csquotes}

% \usepackage{unicode-math}
% \setmainfont{Minion Pro}
% \setmathfont{Latin Modern Math}
% \setmonofont{Iosevka Slab}

\usepackage{tikz}
\usetikzlibrary{cd}
% \usepackage{pgfplots}

\newtheorem{thm}{Theorem}[section]
\newtheorem{cor}[thm]{Corollary}
\newtheorem{lem}[thm]{Lemma}
\newtheorem{prob}[thm]{Problem}
\newtheorem{innercustomthm}{Problem}
\newenvironment{xprob}[1]
  {\renewcommand\theinnercustomthm{#1}\innercustomthm}
  {\endinnercustomthm}

\theoremstyle{definition}
\newtheorem{defn}[thm]{Definition}
\newtheorem{rem}[thm]{Remark}
\newtheorem{exa}[thm]{Example}

\numberwithin{equation}{section}

\newcommand{\comment}[1]{}
\newcommand{\uncomment}[1]{#1}

\newenvironment{solu}{\begin{proof}[Solution]}{\end{proof}}

% notations
\DeclarePairedDelimiter\ceil{\lceil}{\rceil}
\DeclarePairedDelimiter\floor{\lfloor}{\rfloor}

\newcommand{\uto}{\stackrel{u}{\to}}
\newcommand{\norm}[1]{\left\lVert#1\right\rVert}
\newcommand{\abs}[1]{\left|#1\right|}
\newcommand*{\mybox}[1]{\framebox{#1}}
\renewcommand\enquote[1]{\foreignquote{french}{#1}}

\newcommand\SetForm[2]{\left\{ #1 : #2 \right\}}
\newcommand*\conj[1]{\overline{#1}}
\newcommand*\mean[1]{\bar{#1}}

\DeclareMathOperator{\Log}{Log}
\DeclareMathOperator{\Arg}{Arg}
\DeclareMathOperator{\C}{\mathbb{C}}
\DeclareMathOperator{\R}{\mathbb{R}}
\DeclareMathOperator{\N}{\mathbb{N}}
\DeclareMathOperator{\Q}{\mathbb{Q}}
\DeclareMathOperator{\Z}{\mathbb{Z}}
\DeclareMathOperator{\Zp}{\mathbb{Z}^+}
\DeclareMathOperator{\T}{\mathbb{T}}

\DeclareMathOperator{\cT}{\mathcal{T}}
\DeclareMathOperator{\Mero}{Mer}

\DeclareMathOperator{\Hom}{Hom}
\DeclareMathOperator{\End}{End}
\DeclareMathOperator{\Tr}{Tr}
\DeclareMathOperator{\Ima}{Im}

\DeclareMathOperator{\fin}{fin}
\DeclareMathOperator{\std}{std}
\DeclareMathOperator{\product}{prod}

\DeclareMathOperator{\dist}{dist}

\DeclareMathOperator{\esssup}{esssup}

\DeclareMathAlphabet{\mathcal}{OMS}{cmsy}{m}{n}

\oddsidemargin-1mm
\evensidemargin-0mm
\textwidth6.5in
\topmargin-15mm
%\headsep25pt
\textheight8.75in
\footskip27pt

\pagestyle{headings}

\begin{document}

\title{HoTT reading notes}
\author{Sixuan Lou}
\maketitle

\begin{lem}
  We may inhabit
  \[
    transport : \prod_{(C : \prod_{(x : A)} \mathcal{U})}\prod_{(x,y : A)} \prod_{(p : x =_A y)} C(x) \to C(y)
  \].
\end{lem}

\begin{proof}
  Recall, we have
  \begin{align*}
    ind_{=_A} : \prod_{(C : \prod_{(x,y : A)} (x =_A y) \to \mathcal{U})} \left( \textstyle \prod_{(x : A)} C(x,x,refl_x) \right) \to \prod_{(x,y : A)} \prod_{(p : x =_A y)} C(x,y,p)
  \end{align*}
  hence,
  \begin{align*}
    transport(C, x, y, p) :\equiv ind_{=_{A}}( \lambda x . \lambda y . \lambda \_ .  C(x) \to C(y) , \lambda x . id_{C(x)} , x, y, p )
  \end{align*}
\end{proof}


\begin{xprob}{1.1}
  Given function $f : A \to B$ and $g : B \to C$, define composite $g \circ f :
  A \to C$ and show $h \circ (g \circ f) \equiv (h \circ g) \circ f$.
\end{xprob}
\begin{proof}
  \begin{align*}
    & (g \circ f)(x) :\equiv g(f(x))\\
    h \circ (g \circ f)
    & \equiv_{\eta} \lambda x . (h \circ (g \circ f)) (x)
      \equiv_{\beta,\delta} \lambda x . (h ((g \circ f)(x)))\\
    & \equiv_{\beta,\delta} \lambda x . (h (g (f (x))))
      \equiv_{\beta,\delta^{-1}} \lambda x . (h \circ g)(f(x))\\
    & \equiv_{\beta,\delta^{-1}} \lambda x . ((h \circ g) \circ f)(x)
      \equiv_{\eta^{-1}} (h \circ g) \circ f
  \end{align*}
\end{proof}


\begin{xprob}{1.4}
  Define the recursor $rec_{\N}$ in turns of the iterator $iter_{\N}$ and show
  they are propositionally equal with the induction principle $ind_{\N}$.
\end{xprob}

\begin{proof}
  Put $f :\equiv \lambda pr. (s(\pi_1(pr)), c_s(\pi_1(pr), \pi_2(pr)))$
  \begin{align*}
    & rec_{\N} : \prod_{C : \mathcal{U}} C \to (\N \to C \to C) \to \N \to C\\
    & rec_{\N}(C, c_0, c_s, n) :\equiv \pi_2(iter_{\N}(\N \times C, (0, c_0), f,n))
  \end{align*}
  Let $\widetilde{rec_{\N}}$ denote the regular recursor, let $D :\equiv \lambda
  n. rec_{\N}(C, c_0, c_s, n) =_{\N} \widetilde{rec_{\N}}(C, c_0, c_s, n)$. We
  want to construct a term of type
  \[
    \prod_{(C : \mathcal{U})} \prod_{(c_0 : C)} \prod_{(c_s : \N \to C \to C)}
    \prod_{(n : \N)} D(n)
  \]
  We first construct a term of type
  \[
    \prod_{(C : \mathcal{U})} \prod_{(c_0 : C)} \prod_{(c_s : \N \to C \to C)}
    \prod_{(n : \N)} D'(n)
  \]
  where, $D' :\equiv \lambda n . \pi_1(iter_{\N}(\N \times C, (0, c_0), f, n)) =_{\N}
  n$. This is simple, an inhabitant is simply:
  \begin{align*}
    & ind_{\N}(D', refl_{0},\\
    & \ \lambda n . \lambda (pf : D'(n)) .\\
    & \quad ind_{=_{\N}}(E, refl_{s(n)}, s(\pi_1(iter_{\N}(\N \times C,
    c_0, f, n))), s(n), pf))
  \end{align*}
  where $E :\equiv \lambda x . \lambda y . s(x) =_{\N} s(y)$. That term is kind
  of long, we name it as $foo$. Now, with $foo$ in hand, we are able to
  complete our task. The inhabitant is too long to write down, but there is the
  skeleton:

  \[
    ind_{\N}(D, refl_{c_0}, \lambda n . \lambda (pf : D(n)) . g(n,pf))
  \]

  Where $g$ has type:
  \begin{align*}
    & \prod_{n : \N} D(n) \to D(s(n))\\
    & \equiv \prod_{n : \N} rec_{\N}(C, c_0, c_s, n) =_{\N} \widetilde{rec_{\N}}(C, c_0, c_s, n)\\
      & \qquad \to rec_{\N}(C, c_0, c_s, s(n)) =_{\N} \widetilde{rec_{\N}}(C, c_0, c_s, s(n))\\
    & \equiv \prod_{n : \N} rec_{\N}(C, c_0, c_s, n) =_{\N} \widetilde{rec_{\N}}(C, c_0, c_s, n)\\
      & \qquad \to \pi_2(iter_{\N}(\N \times C, (0, c_0), f, s(n))) =_{\N} \widetilde{rec_{\N}}(C, c_0, c_s, s(n))\\
    & \equiv \prod_{n : \N} rec_{\N}(C, c_0, c_s, n) =_{\N} \widetilde{rec_{\N}}(C, c_0, c_s, n)\\
      & \qquad \to \pi_2(f(iter_{\N}(\N \times C, (0, c_0), f, n))) =_{\N} \widetilde{rec_{\N}}(C, c_0, c_s, s(n))\\
    & \equiv \prod_{n : \N} rec_{\N}(C, c_0, c_s, n) =_{\N} \widetilde{rec_{\N}}(C, c_0, c_s, n)\\
    & \qquad \to c_s( \textcolor{blue}{\pi_1(iter_{\N}(\N \times C, (0, c_0), f, n))},
      \textcolor{red}{\pi_2(iter_{\N}(\N \times C, (0, c_0), f, n))})
      =_{\N} c_s(n, \widetilde{rec_{\N}}(C, c_0, c_s, n))
  \end{align*}
  Applying identity elimination twice using $pf$ and $foo$ will replace
  $\color{blue}\pi_1(iter_{\N}(\N \times C, (0, c_0), f, n))$ with $n$ and
  $\color{red}\pi_2(iter_{\N}(\N \times C, (0, c_0), f, n))$ with $\widetilde{rec_{\N}}(C,
  c_0, c_s, n)$. We have the result.
\end{proof}


\begin{xprob}{1.5}
  Show that if we define $A + B :\equiv \Sigma_{x : 2} rec_2(\mathcal{U}, A, B,
  x)$, then we can give a definition of $ind_{A + B}$ such that the definitional
  equalities stated in 1.7 hold.
\end{xprob}

\begin{proof}
  We define $inl(a) :\equiv (0,a)$ and $inr(b) :\equiv (1,b)$. By the definition
  of the induction principle of $\Sigma$-type,
  \begin{align*}
    & ind_{A + B} \equiv ind_{\Sigma_{(x : 2)} rec_2(\mathcal{U}, A , B, x)}\\
    & \qquad : \prod_{(C : \Sigma_{(x : 2)} rec_2(\mathcal{U}, A , B, x) \to \mathcal{U})}
    \left( \prod_{(x : 2)} \prod_{(e : rec_2(\mathcal{U}, A, B, x))} C(x,e) \right)
    \to \prod_{(pr : \Sigma_{(x : 2)} rec_2(\mathcal{U}, A , B, x))} C(pr)\\
    & ind_{A + B}(C, f, (a,b)) :\equiv f(a)(b)
  \end{align*}
  Hence, if we define the regular induction principle in turns of this weird
  definition by:
  \begin{align*}
    & \widetilde{ind_{A + B}}(C, g_0, g_1, inl(a)) :\equiv ind_{A + B}(C, \lambda
    (\_: 2) . g_0, (0, a))\\
    & \widetilde{ind_{A + B}}(C, g_0, g_1, inl(b)) :\equiv ind_{A + B}(C, \lambda
    (\_: 2) . g_0, (1, b))
  \end{align*}
  It's easy to verify everything works out.
\end{proof}

\begin{xprob}{1.6}
  Show that if we define $A \times B :\equiv \prod_{(x : 2)} rec_2(\mathcal{U},
  A, B, x)$, then we can give a definition of $ind_{A \times B}$ for which the
  definitional equalities stated in 1.5 hold propositionally.
\end{xprob}

\begin{proof}
  We define $(a,b) :\equiv \lambda x . rec_2(rec_2(\mathcal{U}, A, B, x), a, b, x)$.
  By the definition of the induction principle of the $\prod$-type, $ind_{A
    \times B}$ is simply the application of a lambda with type $\prod_{(x : 2)}
  rec_2(\mathcal{U}, A, B, x)$ to an argument of type $2$. More precisely,
  \[
    ap : \left( \prod_{(x : 2)} rec_2(\mathcal{U}, A, B, x) \right) \to 2 \to
    rec_2(\mathcal{U}, A, B, x)
  \]
  To recover the usual induction principle, we first attempt to define the
  following term:
  \begin{align*}
    & \widetilde{ind_{A \times B}}' : \prod_{C : A \times B \to \mathcal{U}} \left(\prod_{(x : A)} \prod_{(y : B)} C(x,y) \right) \to \prod_{(pr : A \times B)} C(pr(0), pr(1))\\
    & \widetilde{ind_{A \times B}}'(C, f, g) :\equiv f(g(0))(g(1))
  \end{align*}

  We however, want the body to have type $C(pr)$, not $C(pr(0), pr(1))$, to do
  that, we need to inhabit the propositional uniqueness principle for our new product:
  \begin{align*}
    & uniq_{A \times B} : \prod_{p : A \times B} (p(0), p(1)) =_{A \times B} p\\
    & uniq_{A \times B} :\equiv \lambda p . funext(\lambda (x : 2) . refl_{rec_2(rec_2(\mathcal{U}, A, B, x), p(0), p(1), x)})
  \end{align*}

  Finally,
  \begin{align*}
    & \widetilde{ind_{A \times B}} : \prod_{C : A \times B \to \mathcal{U}} \left( \prod_{(x : A)} \prod_{(y : B)} C(x,y) \right) \to \prod_{(pr : A \times B)} C(pr)\\
    & \widetilde{ind_{A \times B}}(C, f, g)
      :\equiv
      transport( C , (g(0), g(1)), g, uniq_{A \times B}(g))(\widetilde{ind_{A \times B}}'(C, f, g))
  \end{align*}
\end{proof}


\end{document}